%%%%%%%%%%%%%%%%%%%%%%%%%%%%%%%%%%%%%%%%%%%%%%%%%%%%%%%%%%%%%%%%%%%%%%%%%%%%%%%
% CAP�TULO 4
\chapter{DESENVOLVIMENTO DO PROJETO}  
\section{TECNOLOGIAS E FERRAMENTAS}
Para o desenvolvimento da proposta ser�o utilizadas as seguintes tecnologias e ferramentas: 

\subsection{PHP}
O sistema ser� desenvolvido utilizando a linguagem de programa��o PHP, por ser uma linguagem com foco Web e amplamente consolidada, al�m da experi�ncia do autor com a mesma.

\subsection{Laravel}
Ser� utilizado o framework Laravel para o desenvolvimento do sistema, por ser implementado em PHP e oferecer diversos recursos que ser�o usados no projeto.

\subsection{MySQL}
O SGBD escolhido foi o MySQL por sua alta performace e facilidade de encontrar materiais did�ticos. Tambem por ser integrado com o Framework Laravel.

\subsection{Redis}
Para aumentar a performance do sistema e reduzir os custos de consultas no MySQL, ser� utilizado um sistema de cache. O Redis foi escolido por sua alta perfomace, disponibilidade e confiabilidade.

\subsection{Facebook Login}
O Facebook Login permitir� que o sistema acesse informa��es que dif�ceis de coletar com um formul�rio de cadastro, como, as curtidas do usu�rio, o anivers�rio, a cidade natal ou a cidade atual, lista de amigos e lista de fotos. Com esses dados ser� poss�vel gerar uma maior sensa��o de conex�o com o sistema. Logo para realizar a autentica��o com a rede social, ser� utilizado o Facebook Login. 

\section{METODOLOGIA DE DESENVOLVIMENTO}
-- escrever sobre scrum solo.
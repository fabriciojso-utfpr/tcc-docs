\chapter{CRONOGRAMA}
Neste Cap�tulo est� descrito o cronograma para o desenvolvimento desta proposta, sendo dividido em atividades e os meses que cada atividade foi ou ser� realizada, como mostrado na Tabela \ref{cronograma}.
\begin{table}[h]
	\centering
	\caption{Cronograma de Atividades}
	\label{cronograma}
	\begin{tabular}{|c|c|c|c|c|c|c|c|c|c|}
		\hline
		\multirow{2}{*}{\textbf{Atividade}} & \multicolumn{9}{c|}{\textbf{Meses}}                                                                                                                                                                                    \\ \cline{2-10} 
		& Ago                   & Set                   & Out                   & Nov                   & Dez                   & Jan                   & Fev                   & Mar                   & Abr                    \\ \hline
		Defini��o do Tema                   & X                     &                       &                       &                       &                       &                       &                       &                       &                        \\ \hline
		Revis�o Bibliogr�fica               &   X                    & X                     & X                     & X                     & X                     & X                     & X                     & X                     & X                      \\ \hline
		Escrita do Documento                &                       & X                     & X                     & X                     & X                     & X                     & X                     & X                     & X                      \\ \hline
		Levantamento de Requisitos          &                       &                       & X                     & X                     & X                     & X                     &                       &                       &                        \\ \hline
		Diagrama��o do Sistema              &                       &                       &                       &                       &                       & X                     & X                     & X                     &                        \\ \hline
		Implementa��o                       &                       &                       &                       &                       &                       & X                     & X                     & X                     &                        \\ \hline
		Valida��o                           &                       &                       &                       &                       &                       &                       & X                     & X                     &                        \\ \hline
		An�lise dos Resultados              &                       &                       &                       &                       &                       &                       &                       & X                     &                        \\ \hline
		Defesa do TCC                       & \multicolumn{1}{l|}{} & \multicolumn{1}{l|}{} & \multicolumn{1}{l|}{} & \multicolumn{1}{l|}{} & \multicolumn{1}{l|}{} & \multicolumn{1}{l|}{} & \multicolumn{1}{l|}{} & \multicolumn{1}{l|}{} & \multicolumn{1}{l|}{X} \\ \hline
	\end{tabular}
\end{table}


\chapter{Considera��es finais}
Conforme demonstrado nesta proposta, o desenvolvimento do sistema \textit{web} para auxiliar na cria��o, monetiza��o e divulga��o de conte�dos din�micos com �nfase em viraliza��es, torna-se uma importante ferramenta para criadores de conte�dos presentes em redes sociais.  

Com testes a serem executados no \textit{website} www.otariano.com pretende-se validar a plataforma desenvolvida como uma ferramenta para criadores de conte�dos. O site em quest�o implementa, de forma rudimentar, as funcionalidades propostas por esse site atrav�s de plugins personalizados para o  \textit{WordPress}\footnote{https://wordpress.org/}. Com este trabalho, busca-se a consolida��o de uma plataforma escal�vel para possibilitar a monetiza��o de conte�dos a serem publicados em redes sociais.

%Dado todo este cen�rio, o objetivo do desenvolvimento desse sistema � a 
%- O QUE ESPERA ATINGIR COM O DEV DESSE TRABALHO;
%- POSSIVEIS DIFICULDADES
%	- LIMITES e RESTRI��ES
% - RESULTADOS ESPERADOS (QUERO ATINGIR ... )
% - SUBSTITUIR PROJETO ...
% - DADO O CENARIO, PROJETOS RELACIONADOS, A NECESSIDADE DA ARQ PROPOSTA, TEC. DEFINIDAS, PRETEN��O DE DESENVOLVER ... COM OBJETIVO DE ALCAN�AR ... PARA SUBSTITUIR ...